%%%%%%%%%%%%%%%%%%%%%%%%%%%%%%%%%%%%%%%%%%%%%%%%%%%%%%%%%%%%%%%%%%%%%%%%%%%%%%%
\chapter{Summary and Conclusion}~\label{chap:conclusion}
%%%%%%%%%%%%%%%%%%%%%%%%%%%%%%%%%%%%%%%%%%%%%%%%%%%%%%%%%%%%%%%%%%%%%%%%%%%%%%%
%%X1 Conclusion

This thesis presented an approach of scheduling tasks on an asymmetric chip-multiprocessor
with respect to performance requirements and performing a constrained mutation
of the clock speeds of individual processor cores in order to constrain instabilities 
caused due to rapid performance state transitions. 

The procedure and implementation was described and the process was shown to schedule
tasks based on their performance with an acceptable degree of error. The mutation 
method was shown to adapt to the system needs.

Experiments with various static layouts were executed to characterize and showcase 
the performance directed scheduler. Finally, The slowdown and power savings 
achieved with the delta constrained mutator was measured across its entire parameter space
and the best configuration was chosen and compared with two competing mutation methods: 
Ondemand and Conservative. 
The delta constrained mutation method was shown
to give higher returns on power conservation for every percent of performance loss. 