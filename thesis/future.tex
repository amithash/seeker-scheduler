%%%%%%%%%%%%%%%%%%%%%%%%%%%%%%%%%%%%%%%%%%%%%%%%%%%%%%%%%%%%%%%%%%%%%%%%%%%%%%%
\chapter{Future work}~\label{chap:future}
%%%%%%%%%%%%%%%%%%%%%%%%%%%%%%%%%%%%%%%%%%%%%%%%%%%%%%%%%%%%%%%%%%%%%%%%%%%%%%%
%%X1 Future Work

The scheduling methodology can integrate the work presented in \cite{PredictionModel}
for higher accuracy in predicting the performance state required by a particular task. 
Further analysis on the relationship of clock speed invariance of workloads may prove
valuable in improving the select scheduling system and deriving a more optimal methodology
of selecting its threshold values. 
The fact that there exists a correlation between
IPC and speedup achievable with increased clock speed, can be further utilized to produce a more robust system. 
The first steps would be to provide strong support within the Linux kernel for asymmetric
multiprocessors which will enable further study into the aspect of performance
directed scheduling. 

An system call interface was integrated into the existing system which can be utilized 
to develop compiler or runtime frameworks, which would potentially have more information
about program characteristics in developing a hybrid system to provide higher power savings
with lower performance loss. 