%%%%%%%%%%%%%%%%%%%%%%%%%%%%%%%%%%%%%%%%%%%%%%%%%%%%%%%%%%%%%%%%%%%%%%%%%%%%%%%
\chapter{Summary and Conclusion}~\label{chap:conclusion}
%%%%%%%%%%%%%%%%%%%%%%%%%%%%%%%%%%%%%%%%%%%%%%%%%%%%%%%%%%%%%%%%%%%%%%%%%%%%%%%
%%X1 Conclusion

This thesis presented an approach of scheduling tasks on an asymmetric chip-multiprocessor
with respect to performance requirements and performing a constrained system wide
dynamic voltage and frequency transitions in reaction to the demands of the
performance characteristic of the executing workloads. A novel methodology was
presented in controlling the magnitude of these transitions to improve stability
and longevity of the multi-processor system. The methodology was shown to possess 
characteristics to minimize slowdown of workloads while improving the power savings achieved. 

Workloads were shown to possess characteristics to possibly make it immune to clock 
speed improvements thus enhancing the motivation of a performance determined power
management system to conserve power during system active state. Experiments with various 
static layouts were executed to characterize and showcase 
the performance directed scheduler. It was demonstrated that a statically assigned
performance state layout of a system can prove to drastically deteriorate performance 
of the system in an environment with non-deterministic workload sets. 
Two scheduling systems, the ladder and select, were developed to schedule tasks based on
their performance characteristic.

A system was developed to allow a performance directed scheduler to direct an asynchronously
run power optimizer to manage the performance states of each processor in an otherwise 
homogeneous multi-processor environment. The problem was shown to be NP-Hard and an 
efficient algorithm was developed in solving the Multiple-choice knapsack problem 
as a direction based iterative greedy algorithm.

Experiments were conducted to measure slowdown and power savings along the system's entire
parameter space to determine the most favorable constraint on the magnitude of the 
allowed DVFS transitions. This was utilized to compare the power management system
with a popular load based power optimizer, the \textit{ondemand} governor.
The combination of the performance directed scheduler and the delta constrained mutator
was shown to achieve higher power efficiency and better power savings with a marginal 
slowdown.
