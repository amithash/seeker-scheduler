Traditional operating system methodologies in controlling the voltage and 
frequency configuration of the machine are mostly based on the load of the
system. Most methodologies proposed by the research community are either
based on ad-hoc means, thermal emergencies or constraining the power
consumption of the system. None of the methods described consider the 
workload demands which are usually known only to the scheduler. This
thesis presents a methodology of scheduling tasks based on their current
IPC (Instructions per Clock) level on asymmetric multi-processes running
on varied clock speeds. The performance directed scheduler (PDS) maintains 
the workload demand for various performance levels (voltage and frequency configurations).
This is consumed by the mutation engine which is executed on a fixed interval.
The level of mutation of the voltage and frequency configuration is constrained
by a constant factor called the delta constraint. A methodology is proposed
to perform such a demand based constrained mutation. 