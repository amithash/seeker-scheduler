
Traditionally, patterns and repetition in program behavior are
exploited in multiple ways to increase performance in computer
systems.  Cache memory and branch prediction are examples of hardware
technologies that leverage short-term knowledge of program behavior to
enhance performance.  On a different scale, profile-guided compiler
transformations use profile information to customize optimization
decisions to program behavior. However, as the microprocessor industry
shifts to multi-core designs, continued performance gains can only be
achieved by systems that track and adapt to interactions of multiple
concurrent programs.  Multiple cores per system do not universally
improve performance since shared resources and operating conditions
cause interference.  To optimally manage and design multi-core systems
so that they may achieve continued performance gains, new methods of
representing and exploiting program execution behavior are required.

This thesis presents a new representation, the Cardinal Execution Map
(CEM), which describes phase-based program behavior in a compact form.
The representation aids computer system design, helps analyze resource
contention in multi-core environments and directs run-time systems to
better utilize hardware.  The Cardinal Execution Map can be
constructed off-line (statically) and at run time (dynamically) to
enhance the study of microarchitecture, program behaviors, and
multi-core systems.  Presented in the thesis are the execution
modeling framework, a set of analyzes for execution prediction and
program behavior identification, and a multi-context program interaction
modeling system.

